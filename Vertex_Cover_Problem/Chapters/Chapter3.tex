
\chapter{Ειδικές περιπτώσεις} % Main chapter title

\label{Chapter3} % Change X to a consecutive number; for referencing this chapter elsewhere, use \ref{ChapterX}
Το πρόβλημα vertex cover παρόλο που είναι NP-πλήρης στη γενική περίπτωση σε κάποιες ειδικές περιπτώσεις γράφων είναι δυνατό να λυθεί σε πολυωνυμικό χρόνο. Παρακάτω παραθέτονται κάποιες από αυτές τις ειδικές περιπτώσεις και ορισμένες λύσεις και πορίσματα.
%----------------------------------------------------------------------------------------
%	SECTION 1
%----------------------------------------------------------------------------------------

\section{Bigraphs}

\subsection{Εισαγωγικές έννοιες}

\begin{enumerate}
\item
Διμερής γράφοι:\\
Οι διμερείς γράφοι είναι ειδικές περιπτώσεις γράφων που οι κόμβοι τους μπορούν να χωριστούν σε δύο ανεξάρτητα σύνολα $\mathcal{U}$ και $\mathcal{V}$ έτσι ώστε κάθε ακμή να ενώνει ένα κόμβο του $\mathcal{U}$ με έναν κόμβο του $\mathcal{V}$. 
\item
Matching:\\
Δεδομένου ενός γράφου $G=(V,E)$ ονομάζουμε matching $Μ$ στον $G$ ένα σύνολο μη γειτονικών ακμών, δηλαδή ακμές που δεν είναι προσκείμενες στον ίδιο κόμβο. Μέγιστο matching ονομάζουμε το σύνολο με τον μέγιστο αριθμό ακμών.
\end{enumerate}
\begin{figure}[H]
\caption{Matching example}
\centering
\includegraphics{Figures/match.png}\centering
\end{figure}

\begin{figure}[H]
\caption{Maximum matching example}
\centering
\includegraphics{Figures/max_match.png}\centering
\end{figure}

\subsection{Θεώρημα Konig}

Σε αυτό το είδος γράφων βρίσκει εφαρμογή το θεώρημα του Konig το οποίο αναφέρει πως για κάθε διμερή γράφο $G=(V,E)$ ισχύει $\nu(G) = \tau(G)$ όπου\\
$\nu(G) := $ maximum size of a matching in G,\\
$\tau(G) := $ minimum size of a vertex cover in G.

\subsection{Απόδειξη}
\justify
Έστω ένας διμερής γράφος $G = (L, R, E)$ και ένα μέγιστο ταίριασμα $M$. Τότε επειδή κανένας κόμβος ενός vertex cover δεν μπορεί να καλύπτει περισσότερες από δύο ακμές του συνόλου $M$ (διαφορετικά δεν θα ήταν ταίριασμα), ένα vertex cover μεγέθους $|M|$ θα είναι το ελάχιστο vertex cover.\\
Για να δημιουργήσουμε ένα τέτοιο vertex cover, έστω $U$ το σύνολο των μη ταιριασμένων κόμβων του $L$, και $Z$ το σύνολο των ακμών που είτε είναι στο $U$ είτε συνδέονται με αυτό μέσω εναλλακτικών μονοπατιών (alternating paths). Τότε το σύνολο $V' = (L \setminus Z) \cup (R \cap Z)$ είναι ένα vertex cover. 

\subsection{Συμπεράσματα}

Οπότε χρησιμοποιώντας τον αλγόριθμο Hopcroft-Karp, ο οποίος βρίσκει ένα μέγιστο ταίριασμα σε ένα διμερή γράφο σε χρόνο $O(|E| \sqrt{|V|})$, μπορούμε έπειτα να υπολογίσουμε το σύνολο $V'$ αποδοτικά.
Επίσης για όλους τους γράφους ισχύει 
$$ \max_{\text{matching M}} |M| \leq  \min_{\text{vertex cover U}} |U| \leq 2 \cdot \Big( \max_{\text{mathing M}} |M|\Big)$$
\begin{figure}[H]
\caption{Maximum matching - minimum vertex cover example}
\centering
\includegraphics[width=0.4\textwidth]{Figures/KonigTheo.png}\centering
\end{figure}
%----------------------------------------------------------------------------------------
%	SECTION 2
%----------------------------------------------------------------------------------------

\section{Tree graphs}
Για δένδρα υπάρχει ένας άπληστος αλγόριθμος που βρίσκει το ελάχιστο vertex cover σε πολυωνυμικό χρόνο.

\subsection{Εισαγωγικές έννοιες}
Ένα δένδρο είναι ένας μη κατευθυντικός γράφος $G=(V,E)$ που είναι συνεκτικός και δεν έχει κύκλους.

\subsection{Αλγόριθμος}
Η βασική ιδέα του αλγορίθμου είναι η εξής: χρησιμοποιώντας την αναζήτηση πρώτα σε βάθος βρίσκουμε όλα τα φύλλα του δένδρου και έπειτα για κάθε φύλλο επιλέγουμε τον πατέρα του, έπειτα επιλέγουμε κάθε εσωτερικό κόμβο που τα παιδιά του δεν έχουν επιλεχθεί μέχρι να μην υπάρχουν άλλοι κόμβοι να επιλεχθούν.

\begin{figure}[H]
\caption{Vertex cover in tree example}
\centering
\includegraphics[width=0.5\textwidth]{Figures/vc_tree.png}\centering
\end{figure}

%----------------------------------------------------------------------------------------
%	SECTION 3
%----------------------------------------------------------------------------------------

\section{Hypergraphs}
\subsection{Εισαγωγικές έννοιες}
Υπεργράφος είναι μια γενίκευση του γράφου όπου μια ακμή μπορεί να συνδέσει περισσότερους από έναν κόμβους. Πιο αυστηρά ένας υπεργράφος $H$ είναι ένα ζευγάρι $H=(X,E)$ όπου $X$ είναι ένα σύνολο του οποίου τα στοιχεία είναι οι κόμβοι και $E$ είναι ένα σύνολο αποτελούμενο από μη κενά υποσύνολα του $X$ και ονομάζονται υπερακμές. 

Το πρόβλημα εύρεσης του vertex cover ενός hypergraph ονομάζεται transversal, και είναι ισοδύναμο με το hitting set.

%----------------------------------------------------------------------------------------
%	SECTION 4
%----------------------------------------------------------------------------------------

\section{Δυικά προβλήματα}

Το πρόβλημα vertex cover συνδέται και με άλλα συνδυαστικά προβλήματα σε γράφους μέσα από κάποιες δυικές σχέσεις. Κάποια από αυτά τα προβλήματα αναφέρονται παρακάτω.

\subsection{Clique}
Ένα clique ενός μη κατευθυντικού γράφου $G=(V,E)$ είναι ένα υποσύνολο των ακμών $C \subseteq V$ τέτοιο ώστε όλοι οι κόμβοι του να είναι γειτονικοί ανά δύο.\\
Δεδομένου ενός μη κατευθυντικού γράφου $G=(V,E)$ ορίζουμε το συμπλήρωμα του ως $\overline{G} = (V, \overline{E})$ όπου $\overline{E} = \{(u,v):u,v \in{V}, u \neq v, (u,v) \notin{E}\}$, τότε αν υπάρχει ένα clique $C$ στον $\overline{G}$ το $V \setminus C$ είναι ένα vertex cover του $G$.

\begin{figure}[H]
\caption{Clique example}
\centering
\includegraphics[width=0.4\textwidth]{Figures/VertexClique.png}\centering
\end{figure}

\subsection{Independent set}
Το independent set ενός γράφου είναι ένα σύνολο των κόμβων του οι οποίοι δεν είναι γειτονικοί ανά δύο. Το πρόβλημα εύρεσης του μέγιστου independent σετ είναι ένα κλασικό NP-hard πρόβλημα βελτιστοποίησης. Το πρόβλημα αυτό σχετίζεται με το πρόβλημα vertex cover καθώς για κάθε independent set ενός γράφου το συμπλήρωμα του είναι ένα vertex cover για αυτόν τον γράφο.	

\begin{figure}[H]
\caption{Independent set example}
\centering
\includegraphics[width=0.3\textwidth]{Figures/indep_set.png}\centering
\end{figure}
