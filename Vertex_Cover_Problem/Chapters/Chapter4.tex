\chapter{Εφαρμογές} % Main chapter title

\label{Chapter4} % Change X to a consecutive number; for referencing this chapter elsewhere, use \ref{ChapterX}

Το πρόβλημα εύρεσης του vertex cover ενός γράφου βρίσκει εφαρμογές σε διαφόρων ειδών προβλήματα τα οποία μπορεί εκ πρώτης όψεος να μην έχουν κάποιο κοινό γνώρισμα αλλά μπορούν να αναχθούν σε προβλήματα εύρεσης ενός ελαχίστου υποσύνολου το οποίο να "καλύπτει" ένα άλλο σύνολο.

\par{
Κάποια απλά παραδείγματα εφορμογής είναι τα εξής:

Έστω οτί θέλουμε να τοποθετήσουμε τροχονόμους στο οδικό δίκτυο μια πόλης, το πρόβλημα που προκύπτει είναι πώς μπορούμε να τοποθετήσουμε τους τροχονόμους με βέλτιστο τρόπο ώστε να χρησιμοποιήσουμε τον ελάχιστο αριθμό τροχονόμων που να "καλύπτουν" όλους τους δρόμους της πόλης; 
Αυτό το πρόβλημα μπορεί να μοντελοποιηθεί ως ένα minimum vertex cover πρόβλημα όπου οι ακμές είναι οι δρόμοι της πόλης και οι τροχονόμοι οι κόμβοι.
}
\par{
Άλλο παρόμοιο παράδειγμα: έστω οτί θέλουμε να τοποθετήσουμε κάμερες σε ένα κτήριο, πως μπορούμε να τοποθετήσουμε τις κάμερες με βέλτιστο τρόπο ώστε να έχουμε τον ελάχιστο αριθμό καμερών που να καλύπτουν όλους τους χώρους του κτηρίου; Και αυτό είναι ένα πρόβλημα που μπορεί να μοντελοποιηθεί ως minimum vertex cover πρόβλημα.
}

\par{
Ένα άλλο παράδειγμα πρακτικό παράδειγμα είναι στην δυναμική ανίχνευση race conditions. Αν έχουμε ένα νήμα το οποίο γράφει σε μία θέσης μνήμης και έπειτα ένα άλλο νήμα προσπαθήσει να γράψει στην ίδια θέση μνήμης σημαίνει ότι κρατάει ένα lock για συγκεκριμένη θέση. Έτσι μπορούμε να ορίσουμε δύο σύνολα ένα το οποίο περιέχει τα νήματα και ένα που περιέχει τα locks για τις θέσεις μνήμης και ακμές που να αντιπροσωπεύουν την ιδιοκτησία κάποιου lock από ένα νήμα. Τότε το minimum hitting set αντιπροσωπεύει τον ελάχιστο σύνολο απο locks που είναι race-free. Το οποίο χρησιμοποιείται στη συνέχεια για την εξάλειψη περίττων εγγραφών.
}

\par{
	Τέλος ένα παράδειγμα από τον τομέα της υπολογιστικής βιοχημείας όπου σε πολλά προβλήματα χρειάζεται η εξάλειψη συγκρούσεων μετάξυ αλληλουχιών ενός δείγματος. Το πρόβλημα μοντελοποείται ως γράφος όπου οι κόμβοι αντιπροσωπεύουν τις αλληλουχίες του δείγματος και οι ακμές τις μεταξύ τους συγκρούσεις. Ο στόχος είναι να αφαιρεθούν όσον το δυνατόν λιγότεροι κόμβοι ώστε να μην υπάρχουν καθόλου συγκρούσεις στον γράφο.
	}